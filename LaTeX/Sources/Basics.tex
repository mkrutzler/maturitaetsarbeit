%%%%%%%%%%%%%%%%%%%%%%%%%%%%%%%%%%%%%%%%%%%%%%%%%%%%%%%%%%%%%%%%%%%%%%%%
\chapter{Initial Problem}
%%%%%%%%%%%%%%%%%%%%%%%%%%%%%%%%%%%%%%%%%%%%%%%%%%%%%%%%%%%%%%%%%%%%%%%%

The main problem we face can be simplified down into a really easy to formulate but hard to answer question.
How can we order the queued tasks, so that they run in the most optimal way?
What optimal really means is a whole discussion itself not to mention search for the perfect algorithm to achieve that "best" solution.
In this chapter I will give you a quick introduction to the main terms.
In addition to that I will also try to make it as easy as possible to understand by using a situation that most of us already suffered through.

Have you ever wondered why so many people buy already prebottled water? 
How they could eat those snack that you despise?
Why they have a whole plate full of milk in their shopping cart?
Maybe these question seem rather sudden they have two things in common:
One, you never get the answer to them and two, which is the relevant part, you ask them while standing in line at a supermarket.
You ask yourself them while staring at the family of five with two shopping carts and while questioning, whether or not you should have queued at another line.
Also wouldn't it be much better if you, with your two items, went before them?
What is the best order to queue up these people?
As you can already see this is the original, simplified question just reformulated.
We are seaching for the best algorithm that arranges people at the supermarket.
Matter of fact let's just call it by its real name: policy.
In CPU-Scheduling we call the "algorithms" policies or disciplines.
A single customer / shopper is called a job or a process.
In the real world a process can be anything from your drivers to the web browser showing cute cat images.
A queue is a line of people waiting to get their items scanned and the cashier is the "CPU".

\begin{figure}[h]
\begin{minted}[mathescape,
    linenos,
    numbersep=5pt,
    gobble=2,
    frame=lines,
    framesep=2mm,
    ]{python}
  # test
  print("Hello World!")
  variable = 1
  print(variable)
\end{minted}
\caption{Python: First in, First out}
\label{code:fifo}
\end{figure}


Here is a reference to \hyperref[code:fifo]{code block}

%%%%%%%%%%%%%%%%%%%%%%%%%%%%%%%%%%%%%%%%%%%%%%%%%%%%%%%%%%%%%%%%%%%%%%%%
\chapter{How do we compare policies}
%%%%%%%%%%%%%%%%%%%%%%%%%%%%%%%%%%%%%%%%%%%%%%%%%%%%%%%%%%%%%%%%%%%%%%%%

%%%%%%%%%%%%%%%%%%%%%%%%%%%%%%%%%%%%%%%%%%%%%%%%%%%%%%%%%%%%%%%%%%%%%%%%
\chapter{Evolving Supermarket}
%%%%%%%%%%%%%%%%%%%%%%%%%%%%%%%%%%%%%%%%%%%%%%%%%%%%%%%%%%%%%%%%%%%%%%%%

%%%%%%%%%%%%%%%%%%%%%%%%%%%%%%%%%%%%%%%%%%%%%%%%%%%%%%%%%%%%%%%%%%%%%%%%
\chapter{Conclusion}
%%%%%%%%%%%%%%%%%%%%%%%%%%%%%%%%%%%%%%%%%%%%%%%%%%%%%%%%%%%%%%%%%%%%%%%%