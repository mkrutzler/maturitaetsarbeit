\chapter{The ABCs of CPU scheduling}

%%%%%%%%%%%%%%%%%%%%%%%%%%%%%%%%%%%%%%%%%%%%%%%%%%%%%%%%%%%%%%%%%%%%%%%%
\section{Initial Problem}
%%%%%%%%%%%%%%%%%%%%%%%%%%%%%%%%%%%%%%%%%%%%%%%%%%%%%%%%%%%%%%%%%%%%%%%%

In this chapter, I will give a quick introduction to the basics of CPU scheduling. We will look at the proper terms and common approaches to solving the problem of how one should schedule the processes waiting to be run.

\subsection{Welcome to Costco}

The main problem we face can be broken down into a really easy to formulate but hard to answer question.
How can we order the queued tasks so that they run in the most optimal order?
What optimal really means is a whole discussion itself (not to mention the search for the perfect algorithm to achieve that desired best solution).
In this chapter I will give you a quick introduction to the main terms.
In addition, I will also try to make the entry into the world of CPU scheduling as easy as possible to understand by using a situation that most of us have already suffered through.

Have you ever wondered why so many people buy bottled water? 
How they can eat those nasty snacks that you despise?
Why they have the whole cart full of milk?
Maybe these questions seem rather sudden, but they have two things in common:
One, you never get them answered, and two, which is the relevant part, you ask them while standing in line at a supermarket.
You ask them while staring at the family of five with two shopping carts and while questioning, whether or not you should have queued at another line.
Also, wouldn't it be much better if you, with your two items, went before them?
What is the best order to queue up these people?
As you can already see, this is the original, simplified question, just reformulated.
We are searching for the best algorithm that arranges people at the supermarket.
As a matter of fact, let's just call it by its real name: policy.
In CPU-Scheduling we call the algorithms policies or disciplines.
A single customer / shopper is called a job or a process.
In the real world, a process can be anything from your drivers to the web browser showing the cute cat images.
A queue is a line of people waiting to get their items scanned and the cashier is the CPU.

\subsection{How We Compare Policies}

Before we dive deeper into the theory, it is important to note that there is a huge difference between fairness and performance.
Fairness usually ensures that everyone receives the same amount of CPU time.
This kind of policy usually leads to a more responsive system, because each job gets a bit of activity every so often.
The response time is defined as:
$$T_{\text{Response}} = T_{\text{First Run}} - T_{\text{Arrival}}$$
As already mentioned the other measurable aspect is "performance".
Performance is usually measured using the turnaround time.
$$T_{\text{Turnaround}} = T_{\text{Completion}} - T_{\text{Arrival}}$$
There is always a tradeoff between performance and responsiveness, because to achieve the best average turnaround time, the tasks can't be interrupted, or else the time of completion is dragged out.
Think of the average turnaround time like the predicted time that you have to wait in line until you finish with the payment of your groceries. If you constantly get interrupted then this time will go up.
To achieve the best fairness the tasks need to be interrupted, because else new processes have to wait until the previous finishes.

The two formulas above are for a single process. To get a feeling of the overall system, we will look at the average. This average highly depends on the system, because the length of the jobs scales with the average turnaround time proportionally.
If on average there are more items for the cashier to scan, then the average time until he is finished will go up.
Therefore we’ll have to look at the measurements for a predetermined set of jobs.
If we do not have the same system, we can not compare the policies.

\subsection{First lines of Code} \label{sec:first-lines-of-code}

During the next few chapters and sections we'll assume that one knows the amount of time that a process will take to finish, also known as burst time.
Burst time is the time that a process has to run until it is finished.
Like bursting out of joy, when you finally finish your weekly shopping.
In a real world this is almost impossible\footnote{Here I mean knowning the burst time and not escaping the supermarket.}, unless you can time travel.
One of the most straightforward policy is called first come, first served.
This is what we usually suffer through in the queue to pay.
The policy disregards how many item one has in the cart.
The only important attribute is the time you arrive.
The sooner the better.
In the world of computer science, the policy is better refered to as \emph{First In, First Out (FIFO)}.
Take a look at listing \ref{code:fifo} for a basic python implementation.

\begin{figure}[h]
\begin{minted}[mathescape,
    linenos,
    numbersep=5pt,
    gobble=2,
    frame=lines,
    framesep=2mm,
    ]{python}
  # FIFO implementation in python
  queue = [] # initialize empty queue

  # Adding a new process
  def add_process(process):
    queue.append(process)

  # Schedule the processes
  while True:
    if new_process == True: # check if there is a new process
        add_process(process)
    
    next = queue.pop(0) # picks next process
    use_resource(next) # uses resources until finished
\end{minted}
\caption{Python: First in, First out}
\label{code:fifo}
\end{figure}

We save the queue of people as a python list.
If any other job joins, it will just get appended to the end.
As for the scheduling itself:
We just loop through the list until everyone is finished.


In the example above we assume that the use\_resource function on line 14 is already written.
Also it is important to mention that the new\_process variable is just a bool, which gets updated if a new process is waiting to join the queue.
In general python is not really meant for low level programming anyways, so take it with a grain of salt.


%%%%%%%%%%%%%%%%%%%%%%%%%%%%%%%%%%%%%%%%%%%%%%%%%%%%%%%%%%%%%%%%%%%%%%%%
\section{Evolving the Supermarket}
%%%%%%%%%%%%%%%%%%%%%%%%%%%%%%%%%%%%%%%%%%%%%%%%%%%%%%%%%%%%%%%%%%%%%%%%

Now that the scene is set, we return the original question. 
How can we reduce our waiting time?
To give an answer, we will first take a step back.
While the previous chapter talked about the initial situation and proper terms, this one will first loosen up some of the rules in order to ensure that the policies are as easy to understand as possible.

\subsection{Shorterst Job First}

As the name already says, the shortest job will be run first.
In order to achieve this, we assume that the cashier knows, who has the least amount of items.
This buyer will be then handeled before anybody else. As you can already tell, this is not really a fair way to treat your customers.
Still if it comes down to pure turnaround time it is in most cases far better\footnote{With the assumption that the tasks arrive at the same time.} than the basic \emph{First In, First Out}.
Waiting for shortest jobs to finish first is far less severe for longer jobs, because they already have a long burst time. However making the short processes wait out the long ones increases the turnaround time by a lot.
For example let's take process A with a burst time of 10 second and process B with 100 seconds. In case of letting A run first we will have $T_A = 10$ and $T_B = 110$.
Therefore, the average turnaround time is:
$$\frac{10 + 110}{2} = \frac{120}{2} = 60 \text{ seconds}$$
In contrast to that if we let process B run first, we will have the turnaround time of 100 and 110. Therefore, the average is:
$$\frac{110 + 100}{2} = \frac{210}{2} = 105 \text{ seconds}$$



\begin{figure}[h]
    \begin{minted}[mathescape,
        linenos,
        numbersep=5pt,
        gobble=2,
        frame=lines,
        framesep=2mm,
        ]{python}
      # Shortest Job First implementation in python
      class Process:
        def __init__(self, burst):
            self.burst = burst

      # Adding a new process
      def add_process(process):
        queue.append(process)

      # Schedule the processes
      while queue != []:
        if new_process == True: # check if there is a new process
            add_process(process)
        
        queue.sort(key=lambda a: a.burst) # sort according to burst time
        next = queue.pop(0) # picks next process
        use_resource(next) # uses resources until finished
    \end{minted}
    \caption{Python: First in, First out}
    \label{code:sjf}
\end{figure}

As already mentionned in the section \ref{sec:first-lines-of-code}, python is not really meant for writing low level code and therefore it should only be looked at as a simple way of expressing logic.
Even though there were some major improvements, the fact remains that if a client is too late then it will have to wait at least  until the previous shortest customer finishes. This can take a long time and it would be in our best interest to give these newcomers a fair chance. 
It would be great if a rescheduling happens, every time someone joins the queue.

\subsection{Upgrading the Cashier}

In this section our cashier receives a significant upgrade: it can save the state of one buyer and switch to another one.
In technical terms a policy that can do that is called a preemtive policy. 
The switching itself is called a content switch.
While previously, if another customer came with just a single item, it would not get the privilege of cutting infront of the line. 
He would only receive the priority boost, if the current person is finished.
However, now with the preemptive capabilities the cashier can just save and put aside the people.
This feature is not only useful if there is someone that has a shorter time to completion.
Imagine that someone forgot to put something into their basket.
Now they desparately send back their kid to get that item. While he returns the cashier can save the state and do a context switch.
While processes wait for I/O, others can run in their place.
This waiting state is called being blocked and the technique of running another process is overlapping.
At this stage it does not really make sense to present some code, because it is just too out of place, however, in theory, one would do the sorting of the processes everytime a job finishes, gets blocked or another process joins.

\subsection{What about Fairness?} \label{sec:rr}

All of the policies we looked at so far disregarded the emotions of humans.
Most of us want to be treated at least as well as others.
It does not matter whether in a supermarket or at home before the computer, we want fairness and responsiveness.
A fair scheduler like the Round Robin does exactly that. 
It is the most simple to implement for a dynamic system, because all it does is goes around and gives every one contestant a bit of CPU time.
The so called quantum or quanta in plural determines how long a single process can run before it gets replaced by another one.
In supermarket terms: How many items get scanned per customer per cycle.
The logical next step would be to make these quanta as small as possible making the system as fair as possible.
This, however, has a heavy drawback.
The context switching itself takes up resources, meaning that after a while the performance degrades, because of too many context switches.
Therefore the goal is find a quantum, which balances responsiveness and fairness with the performance loss.



 \subsection{Conclusion}

During this chapter we learned the basic methodology of scheduling.
The next step is to move to more sophisticated policies, which act as a whole system rather than just as a solution to a specific problem.
We will also need to figure out the burst time.
To be more precise, we need to figure out how to avoid using it.
Most of the above mentioned policies, except the Round Robin, rely on the fact that we know, when one process finishes.
Take \emph{Shortest Job First} for example: How can we order the processes according to the time to finish if we don't even know when they will finish?
