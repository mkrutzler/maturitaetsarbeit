\documentclass{mimosis}

\usepackage{metalogo}

%%%%%%%%%%%%%%%%%%%%%%%%%%%%%%%%%%%%%%%%%%%%%%%%%%%%%%%%%%%%%%%%%%%%%%%%
% Some of my favourite personal adjustments
%%%%%%%%%%%%%%%%%%%%%%%%%%%%%%%%%%%%%%%%%%%%%%%%%%%%%%%%%%%%%%%%%%%%%%%%
%
% These are the adjustments that I consider necessary for typesetting
% a nice thesis. However, they are *not* included in the template, as
% I do not want to force you to use them.

% This ensures that I am able to typeset bold font in table while still aligning the numbers
% correctly.
\usepackage{etoolbox}

%%%%%%%%%%%%%%%%%%%%%%%%%%%%%%%%%%%%%%%%%%%%%%%%%%%%%%%%%%%%%%%%%%%%%%%%
% Hyperlinks & bookmarks
%%%%%%%%%%%%%%%%%%%%%%%%%%%%%%%%%%%%%%%%%%%%%%%%%%%%%%%%%%%%%%%%%%%%%%%%

\usepackage[%
  colorlinks = true,
  citecolor  = RoyalBlue,
  linkcolor  = RoyalBlue,
  urlcolor   = RoyalBlue,
  unicode,
  ]{hyperref}

\usepackage{bookmark}
%
%%%%%%%%%%%%%%%%%%%%%%%%%%%%%%%%%%%%%%%%%%%%%%%%%%%%%%%%%%%%%%%%%%%%%%%%%
%% Bibliography
%%%%%%%%%%%%%%%%%%%%%%%%%%%%%%%%%%%%%%%%%%%%%%%%%%%%%%%%%%%%%%%%%%%%%%%%%
%%
%% I like the bibliography to be extremely plain, showing only a numeric
%% identifier and citing everything in simple brackets. The first names,
%% if present, will be initialized. DOIs and URLs will be preserved.
%
%\usepackage[%
%  autocite     = plain,
%  backend      = biber,
%  doi          = true,
%  url          = true,
%  giveninits   = true,
%  hyperref     = true,
%  maxbibnames  = 99,
%  maxcitenames = 99,
%  sortcites    = true,
%  style        = numeric,
%  ]{biblatex}
%
%
%%%%%%%%%%%%%%%%%%%%%%%%%%%%%%%%%%%%%%%%%%%%%%%%%%%%%%%%%%%%%%%%%%%%%%%%
% Fonts
%%%%%%%%%%%%%%%%%%%%%%%%%%%%%%%%%%%%%%%%%%%%%%%%%%%%%%%%%%%%%%%%%%%%%%%%

\ifxetexorluatex
  \usepackage{unicode-math}
  \setmainfont{EB Garamond}
  \setmathfont{Garamond Math}

  % Load some missing symbols from another font.
  \setmathfont{STIX Two Math}[%
    range = {
      \sharp,
      \natural,
      \flat,
      \clubsuit,
      \spadesuit,
      \checkmark
    }
  ]
  \setmonofont[Scale=MatchLowercase]{Source Code Pro}
\else
  \usepackage[lf]{ebgaramond}
  \usepackage[oldstyle,scale=0.7]{sourcecodepro}
  \singlespacing
\fi



%%%%%%%%%%%%%%%%%%%%%%%%%%%%%%%%%%%%%%%%%%%%%%%%%%%%%%%%%%%%%%%%%%%%%%%%
% Ordinals
%%%%%%%%%%%%%%%%%%%%%%%%%%%%%%%%%%%%%%%%%%%%%%%%%%%%%%%%%%%%%%%%%%%%%%%%

\makeatletter
\@ifundefined{st}{%
  \newcommand{\st}{\textsuperscript{\textup{st}}\xspace}
}{}
\@ifundefined{rd}{%
  \newcommand{\rd}{\textsuperscript{\textup{rd}}\xspace}
}{}
\@ifundefined{nd}{%
  \newcommand{\nd}{\textsuperscript{\textup{nd}}\xspace}
}{}
\makeatother

\renewcommand{\th}{\textsuperscript{\textup{th}}\xspace}

%%%%%%%%%%%%%%%%%%%%%%%%%%%%%%%%%%%%%%%%%%%%%%%%%%%%%%%%%%%%%%%%%%%%%%%%
% Incipit
%%%%%%%%%%%%%%%%%%%%%%%%%%%%%%%%%%%%%%%%%%%%%%%%%%%%%%%%%%%%%%%%%%%%%%%%

\title{{CPU-Scheduling}}
\subtitle{What waiting in line at the supermarket and your computer have in common}
\author{Mark Krutzler}


%%%%%%%%%%%%%%%%%%%%%%%%%%%%%%%%%%%%%%%%%%%%%%%%%%%%%%%%%%%%%%%%%%%%%%%%
% Code Blocks
%%%%%%%%%%%%%%%%%%%%%%%%%%%%%%%%%%%%%%%%%%%%%%%%%%%%%%%%%%%%%%%%%%%%%%%%

\usepackage{minted}

\usepackage{hyperref}

% SVG support
\usepackage{svg}

\begin{document}

\frontmatter
  \begin{titlepage}
    \vspace*{5cm}
    \makeatletter
    \begin{center}
      \textsc{Kantonsschule Im Lee, Winterthur\\Maturitätsarbeit HS 2024/25}\\
      \vspace*{1cm}
      \begin{Huge}
        \@title
      \end{Huge}\\[0.1cm]
      %
      \begin{Large}
        \@subtitle
      \end{Large}\\
      %
      \emph{by}
      \@author\\
      \emph{under the supervision of}
      Thomas Graf\\
      %
      \vfill
      06. January 2025, Winterthur
    \end{center}
    \makeatother
\end{titlepage}
  
\newpage
\null
\thispagestyle{empty}
\newpage
  

  \begin{center}
  \textsc{Abstract}
\end{center}
%
\noindent
%
Abstract here

  \tableofcontents

\mainmatter

  \part[The ABC's of CPU scheduling]{%
    The ABC's of CPU scheduling\\
    %
    \vspace{1cm}
    %
    \begin{minipage}[l]{\textwidth}
    %
    \textnormal{%
      \normalsize
      %
      \begin{singlespace*}
        \onehalfspacing
        %
        In this chapter I will give a quick introduction to the basics of CPU scheduling. We will look at the proper terms and common approaches to solving the problem of how one should schedule the processes waiting to be run.
      \end{singlespace*}
    }
    \end{minipage}
  }

  %%%%%%%%%%%%%%%%%%%%%%%%%%%%%%%%%%%%%%%%%%%%%%%%%%%%%%%%%%%%%%%%%%%%%%%%
\chapter{Initial Problem}
%%%%%%%%%%%%%%%%%%%%%%%%%%%%%%%%%%%%%%%%%%%%%%%%%%%%%%%%%%%%%%%%%%%%%%%%

\section{Welcome to Costco}

The main problem we face can be broken down into a really easy to formulate but hard to answer question.
How can we order the queued tasks, so that they run in the most optimal order?
What optimal really means is a whole discussion itself not to mention search for the perfect algorithm to achieve that desired best solution.
In this chapter I will give you a quick introduction to the main terms.
In addition, I will also try to make it as easy as possible to understand by using a situation that most of us have already suffered through.

Have you ever wondered why so many people buy prebottled water? 
How they can eat those nasty snack that you despise?
Why they have the whole cart full of milk?
Maybe these question seem rather sudden they have two things in common:
One, you never get the answer to them, and two, which is the relevant part, you ask them while standing in line at a supermarket.
You ask yourself them while staring at the family of five with two shopping carts and while questioning, whether or not you should have queued at another line.
Also wouldn't it be much better if you, with your two items, went before them?
What is the best order to queue up these people?
As you can already see this is the original, simplified question just reformulated.
We are seaching for the best algorithm that arranges people at the supermarket.
Matter of fact let's just call it by its real name: policy.
In CPU-Scheduling we call the algorithms policies or disciplines.
A single customer / shopper is called a job or a process.
In the real world a process can be anything from your drivers to the web browser showing cute cat images.
A queue is a line of people waiting to get their items scanned and the cashier is the CPU.

\section{How we compare policies}

Before we dive deeper into theory, it is important to note that there is a huge difference between fairness and performance.
Fairness usually makes sure that everyone recieves the same amount of CPU time.
This kind of policy turns out to be cyclic, which usually leads to a more responsive system, because each job gets a bit of activity every so often.
On the other hand if we want to optimize for performance we should look at the so called average turnaround time.
Turnaround time is nothing more than the time that a person has to stand in line.
Therefore the average of it is the average waiting time until you exit the supermarket.

\section{First lines of code}

During the next few chapters and sections we'll assume that one knows the amount of time that a process will take to finish, also known as burst time.
Like bursting out of joy, when you finally finish the weekly shopping.
In a real world this is almost impossible\footnote{Here I mean knowning the burst time and not escaping the supermarket.}, except if you can time travel.
With this and setting aside the optimization part, one of the most straightforward policy is called first come, first served.
This is what we usually suffer through in the queue to pay.
The policy disregards how many item one has in the cart.
The only important part is the time you arrive.
The sooner the better.
In the world of computer science the policy is better refered to as First In, First Out (FIFO).
Here is a \hyperref[code:fifo]{simple python implementation}.

We save the queue of people as a python list.
If any other job joins, it will just get appended to the end.
As for the scheduling it self:
We just loop through the list until everyone is finished.

\begin{figure}[h]
\begin{minted}[mathescape,
    linenos,
    numbersep=5pt,
    gobble=2,
    frame=lines,
    framesep=2mm,
    ]{python}
  # FIFO implementation in python
  queue = [] # initialize empty queue

  # Adding a new process
  def add_process(process):
    queue.append(process)

  # Schedule the processes
  while queue != []:
    if new_process == True: # check if there is a new process
        add_process(process)
    
    next = queue.pop(0) # picks next process
    use_resource(next) # uses resources until finished
\end{minted}
\label{code:fifo}
\caption{Python: First in, First out}
\end{figure}
In this example we just assume that the use\_resource function on line 14 is already written.
Also it is important to mention that the new\_process variable is just a bool, which gets update if a new process is waiting to join the queue.


%%%%%%%%%%%%%%%%%%%%%%%%%%%%%%%%%%%%%%%%%%%%%%%%%%%%%%%%%%%%%%%%%%%%%%%%
\chapter{Evolving Supermarket}
%%%%%%%%%%%%%%%%%%%%%%%%%%%%%%%%%%%%%%%%%%%%%%%%%%%%%%%%%%%%%%%%%%%%%%%%





%%%%%%%%%%%%%%%%%%%%%%%%%%%%%%%%%%%%%%%%%%%%%%%%%%%%%%%%%%%%%%%%%%%%%%%%
\chapter{Conclusion}
%%%%%%%%%%%%%%%%%%%%%%%%%%%%%%%%%%%%%%%%%%%%%%%%%%%%%%%%%%%%%%%%%%%%%%%%


  \part[Advanced Policies]{%
    Advanced Policies\\
    %
    \vspace{1cm}
    %
    \begin{minipage}[l]{\textwidth}
    %
    \textnormal{%
      \normalsize
      %
      \begin{singlespace*}
        \onehalfspacing
        %
        After looking at the Basics it is time to dive deeper. Here we will learn about policies that do not rely on the burst time to function. This means that technically they can be used in a real world application. In reality however they are usually first modified to fit the environment and the needs better.
      \end{singlespace*}
    }
    \end{minipage}
  }

  \chapter{Predicting the Future}

\section{Idea}

The first policy we look at is formerly called Multi-Level Feedback Queue or MLFQ short. The create Fernando J. Corbató recieved a Turning Award for it in 1990. 
As the title already says, this policy tries to predict the future behaviour of the processes based on the past.
A job can be generally act in two ways.
Either it is a resources intensive crunching problem (think about exporting a video or compiling code) or it is a program, which needs quick response time (think about your text editor).
In reality most jobs jump between these two states.
We usually want to give the response based processes priority, because that is what the user interacts with and it is here that he primarily feels a delay.

The policy is based on multiple queues, which have each different priorities.
Each process gets assigned to a queue. There are however not set in stone
Based on the reasons above we want to assume that a new process is responsive, because in the worst case scenario, we need to just demote the non-responsive ones.
If however a process turns out to be interactive, than the user does not feel any lag. 
Each process can run a certain amount of time (also called allotment time) before it is deemed as unworthy of the current priority.
If the allotment is used up the process get demoted into a queue below.

\begin{figure}[h]
    \centering
    \includegraphics[width=0.7\textwidth]{Assets/MLFQ-Example-1.pdf}
    \caption{Simple working of MLFQ}
    \label{fig:mlfq-example-1}
\end{figure}


As you can see in figure \ref{fig:mlfq-example-1} the process one gets demoted after a while.
Also the lower the queue is, the higher the allotment time. 
This is because we hope that all of the responsive focused finish before demoting.
Once these are filtered out we have only resource heavy tasks left.
These require more time anyways, so the allotment time is stretched out.

\section{Multiple Processes}

What happens if we introduce another process?
Well, it depends on the priorities. 
Higher priority recieves the CPU time.
If they are on the same queue, than they run using Round Robin, see section \ref{sec:rr}.


\begin{figure}[h]
    \centering
    \includegraphics[width=0.7\textwidth]{Assets/MLFQ-Example-2.pdf}
    \caption{Running multiple processes in MLFQ}
    \label{fig:mlfq-example-2}
\end{figure}

As you can see in figure \ref{fig:mlfq-example-2} once process two is introduced process one is temporarily starved. The problem is solved once they land on the same queue.
There they run alternately.
Still the more tasks we introduce the more prevalent the starving issue gets.

\begin{figure}[h]
    \centering
    \includegraphics[width=0.7\textwidth]{Assets/MLFQ-Example-3.pdf}
    \caption{Starving Processes in MLFQ}
    \label{fig:mlfq-example-3}
\end{figure}

\section{Solving the Starving Problem}

Solving starvation is actually easy.
All we have to make sure that once in a while everybody gets their deserved CPU time.
To do that we introduce a priority boost. All it does is just put every process into priority one every so often.

\begin{figure}[h]
    \centering
    \includegraphics[width=0.7\textwidth]{Assets/MLFQ-Example-4.pdf}
    \caption{Solving Starvation in MLFQ}
    \label{fig:mlfq-example-4}
\end{figure}

\section{Wrapping Up}

This chapter showed us, how one could approach a scheduling policy that works without needing to know the burst time.
In the end we still end up with many variable like the allotment time per queue, the amount of queues, the quanta for the Round Robin and others.
If used in a real system the best approach would be to set some sensible defaults and let the administrator to adjust the parameters when needed.

\chapter{Using Lottery to Schedule Processes}

% This ensures that the subsequent sections are being included as root
% items in the bookmark structure of your PDF reader.
\bookmarksetup{startatroot}
\backmatter

\chapter{Sources}
\begin{itemize}
    \item \url{https://github.com/Pseudomanifold/latex-mimosis}
    \item \url{https://en.wikipedia.org/wiki/Binary_search}
    \item \url{https://www.quora.com/Why-is-look-up-faster-for-a-Binary-Tree-than-a-Linked-List}
\end{itemize}


\end{document}
