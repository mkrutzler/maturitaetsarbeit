\documentclass{report}

\usepackage{titlesec}
\usepackage{titling}
\usepackage{hyperref}
\usepackage[margin=1in]{geometry}
\usepackage[normalem]{ulem}
\usepackage{xcolor}
\usepackage{graphicx}
\usepackage[framemethod=TikZ]{mdframed}
\usepackage{etoolbox}

% sets up mdframed
\newcounter{defi}[section]\setcounter{defi}{0}
\renewcommand{\thedefi}{\arabic{section}.\arabic{defi}}
\newenvironment{defi}[2][]{%
\refstepcounter{defi}%
\ifstrempty{#1}%
{\mdfsetup{%
frametitle={%
\tikz[baseline=(current bounding box.east),outer sep=0pt]
\node[anchor=east,rectangle,fill=blue!20]
{\strut Definition~\thedefi};}}
}%
{\mdfsetup{%
frametitle={%
\tikz[baseline=(current bounding box.east),outer sep=0pt]
\node[anchor=east,rectangle,fill=blue!20]
{\strut Definition~\thedefi:~#1};}}%
}%
\mdfsetup{innertopmargin=10pt,linecolor=blue!20,%
linewidth=2pt,topline=true,%
frametitleaboveskip=\dimexpr-\ht\strutbox\relax
}
\begin{mdframed}[]\relax % Added this line
}{%
\end{mdframed} % Added this line
}

% Example usage of mdframed
% \begin{defi}[Pythagoras' theorem]{defi:pythagoras}
% In a right triangle, the square of the hypotenuse is equal to the sum of the squares % of the catheti.
% $$a^2+b^2=c^2$$
% \end{defi}


%define "squiggly" to make red squiggly underline
\makeatletter
\def\squiggly{\bgroup \markoverwith{\textcolor{red}{\lower3.5\p@\hbox{\sixly \char58}}}\ULon}
\makeatother


% Stuff to edit.
\title{CPU-Scheduling}
\newcommand{\thesubtitle}{The Illusion of Multitasking}
\newcommand{\currentdate}{2024/25}
\author{Mark Krutzler}
\newcommand{\auinstitution}{Kantonschule Im Lee}

\titleformat{\section}
{\huge\bfseries}
{\thesection}
{0.5em}
{}[\titlerule]

\titleformat{\subsection}
{\bfseries\LARGE}
{\thesection}
{0.5em}
{}

\titleformat{\subsubsection}
{\large\bfseries}
{\thesection}
{0.5em}
{}


\renewcommand{\maketitle}{
\begin{center}


{\Huge\bfseries
\thetitle}
\vspace{0.5em}\\
{\LARGE\thesubtitle}


{\rule{0.4\textwidth}{.4pt}}

{\bfseries \theauthor}\\
\textit{\auinstitution \ --- \currentdate}
\end{center}
}

\begin{document}

\begin{titlepage}
\vspace*{\fill}
\centering
\maketitle
\vspace*{\fill}
\end{titlepage}

\tableofcontents



\part{Introduction / Forewords}

- what is CPU scheduling
- why do we need it
    - how does a CPU work / why isnt there multitasking
- What is the goal of my project
- bit of an overview over the whole report



\part{The ABCs of CPU Scheduling}

\chapter{Introduction to Part 1}



\section{Metrics}

    - turnaround time (performance)
    - response time (performance)
    - fairness

\begin{defi}[Turnaround Time]{dfn:turnaround}
The turnaround time is ...\\
It is calculated as follows:
$$T_{Turnaround} = T_{Completion} - T_{Arrival}$$
\end{defi}

\chapter{Basic Algorithms}

\section{Optimized for Performance}

\subsection{First In, First Out (FIFO )}

\subsection{Shortest Job First (SJF)}

\subsection{Shortest Time-to-Completion (STCF)}

\section{Optimized for Fairness}

\subsection{Round Robin (RR)}





\part{The Industry Standard of CPU-Scheduling}

\chapter{Multi-Level Feedback Queue}

\chapter{Lottery and Stride Scheduling}

\part{Examples}

\chapter{My Implementation}

\chapter{Solaris Scheduling}

\chapter{Linux 2.6 Fair Scheduler}



\part{conclusion}


\end{document}
